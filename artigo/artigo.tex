\documentclass[conference]{IEEEtran}
\IEEEoverridecommandlockouts
% The preceding line is only needed to identify funding in the first footnote. If that is unneeded, please comment it out.
\usepackage{cite}
\usepackage{amsmath,amssymb,amsfonts}
\usepackage{algorithmic}
\usepackage{graphicx}
\usepackage{textcomp}
\usepackage{xcolor}
\usepackage[english, brazil]{babel}
\def\BibTeX{{\rm B\kern-.05em{\sc i\kern-.025em b}\kern-.08em
    T\kern-.1667em\lower.7ex\hbox{E}\kern-.125emX}}
\begin{document}

\title{Manutenção de Fatores Biológicos de Crescimento da Orquídea {\itshape{Phalaenopsis}} com Internet das Coisas}


\author{\IEEEauthorblockN{Leonardo Miranda Amaral}
\IEEEauthorblockA{\textit{Dept. Engenharia de Computação} \\
\textit{Centro Universitário IESB}\\
Brasília, Distrito Federal, Brasil}
\and
\IEEEauthorblockN{Felipe Rodrigues Araújo}
\IEEEauthorblockA{\textit{Dept. Engenharia de Computação} \\
\textit{Centro Universitário IESB}\\
Brasília, Distrito Federal, Brasil}
\and
\IEEEauthorblockN{Rafael Roriz Barbosa}
\IEEEauthorblockA{\textit{Dept. Ciência da Computação} \\
\textit{Centro Universitário IESB}\\
Brasília, Distrito Federal, Brasil}
\and
\IEEEauthorblockN{Daniel Anes Da Silva Almeida}
\IEEEauthorblockA{\centerline{\textit{ Dept. Ciência da Computação}} \\
\textit{Centro Universitário IESB}\\
Brasília, Distrito Federal, Brasil}
}

\maketitle

\selectlanguage{brazil}
\begin{abstract}
Tendo em vista a necessidade da presença e auxílio de pessoas, sejam jardineiros ou responsáveis pelas plantas e jardins, nota-se a necessidade da construção de sistemas que possam automatizar e contribuir para a manutenção dos fatores pré-colheita. O presente trabalho propõe-se a criar um sistema com Internet das Coisas, API e Aplicação móvel para manutenção dos fatores pré-colheita do objeto de estudo. 

\textbf{PALAVRAS CHAVE: Biologia, Jardim Inteligente, Microcontrolador, IoT, Sensores}.
\end{abstract}

\section{Introdução}

Automação é a denominação dada para sistemas automáticos de controle, pelos quais os mecanismos verificam seu próprio funcionamento, efetuando medições e introduzindo correções sem a necessidade da interferência humana. Atualmente, a automação está presente em diferentes níveis de atividades do homem, desde as residências até processos industriais.

A internet das coisas, ou como mais conhecida do inglês, Internet of Things (IOT), é a técnica de se utilizar tecnologias embarcadas em objetos do dia a dia, como utensílios domésticos e equipamentos eletrônicos em geral, conectando-os à internet e trocando informações com outros objetos, usuários e servidores, com vistas a determinado objetivo.

Tal plataforma permite impactar positivamente no uso dos recursos de monitoramento e automação aplicados no jardim, que geralmente consiste em um microcontrolador central ao qual outros objetos estão conectados. O jardim inteligente consiste em um microcontrolador como hub para o qual diferentes tipos de sensores, como sensor de umidade, sensor temperatura, tanto do solo como do ar, e bomba d’água para irrigação.

Quando as pessoas tentam fazer plantações e montar seu próprio jardim, tais plantas como a Orquídea Phalaenopsis que é objeto de pesquisa podem demandar cuidados especiais e constantes nos seus estágios iniciais ou durante toda a vida da planta. Com o passar dos dias, devido à falta de manutenção, ter a planta destruída.

O microcontrolador enviará os dados via o serviço MQTT Cloud Service para uma API REST. Um aplicativo de computação móvel fará requisição dos dados da API REST utilizando protocolo HTTP. Tal aplicativo de computação móvel vai fornecer ao usuário as informações de temperatura e umidade, e o controle do sistema de automação de manutenção da planta.

Este protótipo ajudará pessoas para monitorar automaticamente os parâmetros e garantir a manutenção do jardim, desempenhando um papel vital e servindo como um bom companheiro para as plantas.

Este artigo está organizado da seguinte forma: Seção 1 a 3: Apresentação de uma abordagem de tratamento de plantas com base nas tecnologias apresentadas. Seção 4 a 5: Objetivos e formas de aplicação do projeto. Seção 6: Metodologia e estudos da planta objeto de pesquisa Orquídea Phalaenopsis, e documentação dos métodos utilizados. Seção 7: Trabalhos referências. Seção 8: Expectativa de resultados as serem obtidos no trabalho e suas conclusões.


\section{Contextualização}

 Na atual era de tecnologia, a automação é a chave para o crescimento de vários setores da economia. Desde manufatura e agricultura até serviços e logística podem ser melhorados por meio da tecnologia. Ao utilizar a internet das coisas que é a técnica de usar poder computacional para monitorar e controlar alguns parâmetros do dia a dia. Novos sistemas estão sendo desenvolvidos fazendo uso de sensores, softwares e protocolos de comunicação para automatizar tarefas específicas. Atualmente, algumas pessoas preferem ter seus próprios jardins ou outras pequenas árvores de frutas em suas residências. Muitas vezes, precisam ser cautelosas durante os estágios iniciais de crescimento.Como resultado, o jardim pode eventualmente ser destruído por uma pequena falta de atenção. Também, condições climáticas podem favorecer o encurtamento da vida do jardim. Pois, algumas plantas podem morrer devido a falta de umidade, falta de incidência solar ou calor intenso.

\section{Problematização}
Devido a planta usada como objeto de estudo demandar elevado cuidado e atenção. Porém, ao faltar estes pode fazer com que a planta morra. Mas quais fatores biológicos  são importantes para compreender como afetam o desenvolvimento da Orquídea {\itshape{Phalaenopsis}}? Como  alguns fatores podem ser supridos sem a necessidade da atenção humana? Como a aplicação pode ser feita de forma segura e que possa ser consumida por diversas outras aplicações? \\

\section{Objetivo geral}
Ao se utilizar o conceito de internet das coisas, o sistema proposto tem como objetivo integrar todos os sensores e componentes para enviar dados em tempo real para o aplicativo de computação móvel do usuário através de uma API, além de armazenar os dados para registro. Com isto, o usuário tem controle de alguns parâmetros essenciais para a sobrevivência de sua Orquídea {\itshape{Phalaenopsis}}, e auxílio para manutenção de seus fatores pré-colheita, assim como auxiliar na otimização dos recursos utilizados para a manutenção da planta, como a água.

\section{Objetivos específicos}
\begin{itemize}
    \item Conhecer os requisitos de umidade, intensidade de luz e temperatura da Orquídea {\itshape{Phalaenopsis}};
    \item Fazer uso de sensores para detectar umidade e temperatura tanto do ar como do solo, e intensidade de luz conectado a um microcontrolador;
    \item Fazer o sistema de controle de umidade conectado ao microcontrolador;
    \item Fazer o sistema de controle de temperatura conectado ao microcontrolador;
    \item Fazer o sistema de controle de luminosidade conectado ao microcontrolador;
    \item Criar uma API REST para coletar dados e armazenar no banco de dados;
    \item Enviar os dados do microcontrolador utilizando o serviço MQTT Cloud Service para a API REST;
    \item Baseado nos parâmetros definidos pelo usuário, a API utiliza o microcontrolador para acionar os sistemas de controle de temperatura e umidade;
    \item O aplicativo de computação móvel faz a requisição desses dados para a API REST utilizando o protocolo HTTP para que sejam fornecidos para o usuário, além de enviar configurações dos sistemas de controle de umidade e temperatura.
\end{itemize}

\section{Referencial teórico}
\subsection{Biologia}
Conhecer os aspectos biológicos da planta contribui para o seu melhor cuidado. A Floricultura, são todas as plantas ornamentais sem ramos lenhosos, incluindo plantas de canteiro e de jardim, anuais ou perenes, flores de corte, ramagens de corte, plantas floríferas em vaso, plantas de folhagem para uso em interiores e material de propagação,\cite{b2} e é um ramo da agricultura em que se lida com o cultivo de flores. A indústria da floricultura a cada dia cresce mais.\cite{b1} Esta abrange múltiplas formas de exploração e cultivo, assim como: flores de corte, flores secas, flores e plantas em vaso, folhagens, mudas, plantas ornamentais, bulbos, tubérculos, rizomas, estacas e sementes.\cite{b2}

O objeto de estudo se encaixa em Flores de Corte, também conhecidas como Cut Flowers. As flores podem ser frescas, secas ou preservadas, vendidas em forma de haste, ramalhetes ou arranjos. \cite{b2}

Contudo, as flores de corte são vulneráveis a altas perdas pós-colheita. O manuseio de pós-colheita é um processo que começa no nível da fazenda, onde se inicia sua plantação. Devido a isto, torna-se importante o conhecimento do manuseio das flores de forma apropriada para mantimento de sua qualidade e frescor. É difícil determinar a vida pós-colheita de uma flor, pois, é influenciada por diversos fatores {\itshape{viz.}}, Fatores pré-colheita, Estágio de colheita e Fatores pós-colheita.\cite{b1}

No entanto, como o foco deste trabalho é a qualidade do mantimento e vida da planta, apenas abordaremos os Fatores pré-colheita. Diversos são os fatores pré-colheita, e podem ser classificados como ambientais e culturais.\cite{b3}

Nos fatores (ou práticas) ambientais pré-colheita se encaixam: temperatura, umidade relativa do ar, luminosidade, textura do solo, vento, altitude e chuva. Nos fatores (ou práticas) culturais se incluem: nutrição mineral, manejo do solo, poda, raleio, aplicações de produtos químicos, uso de porta-enxertos, espaçamento do plantio, irrigação e drenagem.\cite{b3}

O escopo deste trabalho irá abordar os seguintes fatores como objetos de estudo: intensidade de luz, temperatura, e umidade do solo e ar.

\subsubsection{Intensidade de Luz}
A intensidade de luz, também conhecida como luminosidade é um dos fatores mais importantes, pois está relacionada à fotossíntese, e é usada como fonte de energia pelas plantas. A luz ajuda na regulação de diversos processos fisiológicos. A luz também determina a quantidade de carboidratos nas plantas durante o crescimento em que influencia na qualidade de conservação. PLantas que contêm folhas suaves e flexíveis são plantas em que provavelmente são sensíveis a luz, e não deveriam ser colocadas em uma janela com muita radiação solar. A maioria das orquídeas preferem luz indireta ou luz filtrada com sombra. A Orquídea {\itshape{phalaenopsis}} necessita de luzes baixas (1200-2000 f.c.). \cite{b3, b1, b4, b5}

\subsubsection{Temperatura}
Geralmente, altas temperaturas resultam em altos níveis de respiração. A luminosidade e alta temperatura têm relação direta, e não podem ser tratados isoladamente, pois altas temperaturas têm uma forte associação com a exposição direta ao sol. A temperatura é fator determinante para muitos eventos fisiológicos, e está diretamente relacionada às suas propriedades qualitativas, como o conteúdo de açúcares nos frutos. O resfriamento é essencial para reduzir mudanças metabólicas como atividade enzimática e retardar a maturação da flor. \cite{b3, b4, b5} A Orquídea {\itshape{phalaenopsis}} é uma orquídea que pode ser classificada como orquídea de climas quentes ou tropicais, pois, ela necessita de temperaturas entre 24ºC e 30º graus durante o dia e temperaturas de 18ºC e 20ºC durante a noite. \cite{b6, b7}

\subsubsection{Umidade}
A quantidade de vapor de água no ar representa a umidade. Isto ajuda na manutenção da temperatura da planta e no controle da perda de água da superfície da planta. Alta umidade resulta em alta fotossíntese. Em geral, as orquídeas requerem 80-85\% de umidade para um crescimento satisfatório. A maioria das orquídeas preferem água com pH 5.0-6.5. A irrigação com baixo ou alto pH ou elevados níveis de minerais dissolvidos podem dificultar a retenção dos nutrientes. A água da chuva é o ideal. A Orquídea {\itshape{phalaenopsis}} pode ter uma melhor saúde com sua umidade entre 50-70\%. \cite{b1, b4, b5, b6, b7}

\subsection{Hardware}
Além do software, o sistema proposto para se ter conhecimento dos fatores da Orquídea e auxiliar na sua vida possui componentes físicos e eletrônicos necessários para a contribuição ao projeto. A seguir serão discutidos os componentes físicos a serem utilizados.

\subsubsection{Microcontrolador}
Com o passar dos anos os microcontroladores começaram a ficar mais acessíveis e a serem usados para automatização de diversas áreas, como casas inteligentes, dispositivos vestíveis, monitoramento de qualidade do ar e diversas outras aplicações.
O ESP32 foi lançado no mercado em Setembro de 2016 pela Espressif Systems para substituir o ESP8266. O dispositivo ESP32 é um dispositivo poderoso com Wi-Fi e Bluetooth® já embutidos, feito para ser a solução perfeita para dispositivos IoT,\cite{b8} além de trazer elevado custo-benefício devido a já vir com as comunicações sem fio já embutidas. O ESP32 também possui um melhor processador em comparação a outros microcontroladores.\cite{b8}
O ESP32 irá ser utilizado com o Wi-Fi para poder se comunicar com o backend e ocorrer o armazenamento e transferências de dados, além de também se comunicar com a aplicação móvel para ocorrer a troca de informações. É necessário a utilização de sensores em conjunto com o microcontrolador para que possa ser feita a verificação dos parâmetros a serem abordados na planta.

\subsubsection{Sensor de luminosidade}
O sensor a ser utilizado para incidência de luz será o LDR, Light Dependent Resistor. Este é um fotoresistor, sendo feito de materiais semicondutores com alta resistência sensível a luz. A resistência do LDR diminui conforme a incidência de luz aumenta sobre o sensor e aumenta quando a incidência da luz é menor.\cite{b11}

\subsubsection{Sensor de temperatura e umidade do ar}
O sensor de temperatura a ser utilizado será o DHT11. Este é um sensor digital de baixo custo em que verifica-se a temperatura e a umidade do ar. O sensor é feito da seguinte forma: contém uma porção capacitiva sensitiva utilizada para a medição da umidade e um termistor para a detecção da temperatura. \cite{b9, b10}

\subsubsection{Sensor de umidade do solo}
O módulo de sensor de umidade do solo possui dois componentes, YL69 e YL38, em conjunto detectam o nível de umidade presente no solo. Essa medida ocorre com o nível de água presente no solo. É utilizada a propriedade de resistência elétrica do solo. A sensitividade é equivalente à diferença da água no solo; maior a quantidade de água no solo, maior é a condutividade e, como resultado, menor é sensitividade. \cite{b9, b10}

\subsection{Software}
O Software faz-se necessário ao sistema devido a ser o meio usado para controle do microcontrolador e sensores, assim como armazenamento e relatação dos dados obtidos.

\subsubsection{API}
A principal proposta das APIs (Application Programming Interface) é tornar disponível bibliotecas de código, SDKs, e frameworks para programadores. \cite{b12, b13}

Os principais objetivos com o uso da API no sistema proposto são: fazer separação do backend e frontend, trazer possibilidade de consumo da aplicação, uso dos dados e comunicação com o microcontrolador por outras aplicações, descentralizar a aplicação de um único dispositivo e trazer maior disponibilidade de dados em tempo real.

O estilo arquitetural a ser usado para a API é o REST (Representational State Transfer). Nesta, busca-se seguir os seguintes padrões: sem estado (stateless), relacionamento client-server desacoplado, armazenável em cache, interface uniforme, sistema em camadas e opcionalmente prover código sobre demanda. \cite{b12, b13}

Outro padrão a ser utilizado por este projeto é o padrão de design MVC (Model-View-Controller) para a arquitetura da API. Este foi apresentado por Krasner e Pope no sistema Smalltalk \cite{b16}. Essas camadas propõem o seguinte: 
\begin{itemize}
    \item A Model é um objeto ou conjunto de objetos representando dados ou um processo como a base de dados, e processo de máquinas. Gerencia a informação na forma de dados em que é utilizada para representar a saída com auxílio da View. Ou seja, esta camada será utilizada para a lógica da aplicação, regras de negócio e definição de lógica. \cite{b14, b15}
    \item A View é a camada de visualização do estado da model. É utilizada para criar a interface de usuário da aplicação. A interface de usuário quando utilizada é feita para a interação dos usuários com páginas web. \cite{b14, b15}
    \item A Controller é a camada usada para responder à solicitações feitas pelo usuário. É a conexão entre o usuário e o sistema. A Controller lida tanto com a Model como a View. Controla como ocorre o fluxo de dados no modelo e atualiza a visualização assim que que o dado é alterado. \cite{b14, b15}
\end{itemize} 

\subsubsection{Aplicativo móvel}
A aplicação móvel é responsável pela interação do usuário com a API. Na interação desta que é possível a autenticação do usuário, adicionar plantas para serem avaliadas, fazer o controle dos sensores e fatores pré-colheita das plantas a serem avaliadas.


\section{Trabalhos correlatos}
O trabalho introduzido neste artigo é relacionado a Internet das Coisas, Jardim Inteligente. Além de utilizar uma API e uma aplicação móvel. A seguir, será abordado alguns trabalhos utilizados como base para que este possa ser desenvolvido.

\subsection{IoT based smart garden monitoring system using NodeMCU
microcontroller}
Neste, o sistema proposto busca integrar os sensores e componentes para estatísticas em tempo real atuando em jardins. Além de utilizar o Wi-Fi para comunicação com o ESP8266. O dado é transferido direto para a aplicação móvel.  

Os sensores utilizados são: Sensor de temperatura (DS18B20), Sensor de temperatura e umidade(DHT11) e Sensor de umidade do solo (YL-69). \cite{b11}

\subsection{Smart Garden Monitoring and Control System with
Sensor Technology}
O sistema utilizado é um Microcontrolador Arduino UNO, Sensor de temperatura e umidade (DHT11), Sensor de umidade do solo (SM100) e aplicação móvel. O sistema se propõe a controlar e monitorar todos os parâmetros como temperatura, umidade, umidade do solo, e intensidade de luz. \cite{b10}

\section{Resultados esperados}
Através do sistema proposto com o microcontrolador e sensores em conjunto com a aplicação móvel e a API, busca-se a sobrevivência da Orquídea {\itshape{phalaenopsis}} mediante a manutenção dos fatores pré-colheita colocados como objetos de estudo, ou seja, a luminosidade, umidade do solo e do ar, e temperatura. Assim como, espera-se que para cada objetivo específico a seguir tenha seus resultados esperados associados.
\begin{itemize}
    \item Ao conhecer os requisitos de umidade, intensidade de luz e temperatura da Orquídea {\itshape{Phalaenopsis}} espera-se utilizar os parâmetros obtidos para contribuir com os dados a serem utilizados no sistema para manutenção dos fatores pré-colheitas utilizados neste trabalho;
    \item Ao fazer uso de sensores para detectar umidade e temperatura tanto do ar como do solo, e intensidade de luz conectado a um microcontrolador espera-se utilizar os parâmetros obtidos em cada sensor na aplicação móvel para manutenção dos fatores pré-colheitas utilizados neste trabalho e obter o histórico dos dados da planta;
    \item Ao fazer o sistema de controle de umidade conectado ao microcontrolador espera-se obter informações sobre a umidade do solo da planta;
    \item Ao fazer o sistema de controle de temperatura conectado ao microcontrolador espera-se obter informações sobre a temperature e umidade do ar da planta;
    \item Ao fazer o sistema de controle de luminosidade conectado ao microcontrolador espera-se obter informações sobre a incidência solar sobre a planta;
    \item Ao criar uma API REST para coletar dados e armazenar no banco de dados espera-se utilizá-la para comunicação com o microcontrolador assim como também com a aplicação móvel;
    \item Enviar os dados do microcontrolador utilizando o serviço MQTT Cloud Service para a API REST, com isto espera-se que ocorra a comunicação com o microcontrolador e a passagem de dados do microcontrolador para a API;
    \item Baseado nos parâmetros definidos pelo usuário, a API utiliza o microcontrolador para acionar os sistemas de controle de temperatura e umidade, com isto espera-se o funcionamento do sistema físico e manutenção dos fatores pré-colheita da planta;
    \item O aplicativo de computação móvel faz a requisição desses dados para a API REST utilizando o protocolo HTTP para que sejam fornecidos para o usuário, além de enviar configurações dos sistemas de controle de umidade e temperatura, com isto espera-se que ocorra a troca da comunicação entre a API e a aplicação móvel. 
\end{itemize}


\section{Cronograma}

\subsection{Etapa 1 - 30/03/2022}
\begin{itemize}
    \item Pesquisa sobre as necessidades nutricionais e ambientais da orquídea phalaenopsis, e conhecimentos biológicos dos parâmetros a serem utilizados;
    \item Análise de Requisitos API;
    \item Criar páginas de autorização do Aplicativo Móvel.
\end{itemize}

\subsection{Etapa 2 -  06/04/2022}
\begin{itemize}
    \item Escrever Artigo;
    \item Criar protótipo Figma Completo;
    \item Criar rotas de autorização da API REST, login/redefinir senha/registro;
    \item Implementar integração com api de autorização do Aplicativo Móvel.
\end{itemize}

\subsection{Etapa 3 -  13/04/2022}
\begin{itemize}
    \item Apresentação P1.
\end{itemize}

\subsection{Etapa 4 - 27/04/2022}
\begin{itemize}
    \item Instalar sensores no microcontrolador em Arduino;
    \item Implementar rotas de dashboard da API;
    \item Implementar Páginas de Dashboard na  Aplicação Mobile e Integração com a API;
    \item Implementar Módulo de Controle de umidade API;
    \item Implementar Página de Controle de umidade Aplicação Mobile.
\end{itemize}

\subsection{Etapa 5 - 04/05/2022}
\begin{itemize}
    \item Implementar o Controle de umidade ESP 32;
    \item Implementar o Controle de temperatura ESP 32;
    \item Implementar o Módulo de Controle de temperatura API;
    \item Implementar Módulo de Controle de temperatura Aplicação Mobile.
\end{itemize}

\subsection{Etapa 6 - 11/05/2022}
\begin{itemize}
    \item Implementar o Módulo de Controle de luminosidade API;
    \item Implementar Módulo de Controle de luminosidade Aplicação Mobile;
    \item Implementar integração com com usuário ESP 32;
    \item Implementar integração com API;
    \item Implementar Histórico Temperatura.
\end{itemize}

\subsection{Etapa 7 - 18/05/2022}
\begin{itemize}
    \item Implementar Histórico umidade.
\end{itemize}

\subsection{Etapa 8 - 25/05/2022}
\begin{itemize}
    \item Testes.
\end{itemize}

\subsection{Etapa 9 - 01/06/2022}
\begin{itemize}
    \item Revisão.
\end{itemize}


\begin{thebibliography}{00}
\bibitem{b1} Verma, J. and Singh, P.,``Post-harvest Handling and Senescence in Flower Crops: An Overview,'' Agricultural Reviews. 2021.
\bibitem{b2} Augusto Porto Oliveira, Alfredo e Simone de Castro Pereira Brainer, Maria, `` Floricultura: Caracterização e Mercado,'' Documentos do Etene, Escritório Técnico de Estudos Econômicos do Nordeste, Nº 16, Banco do Nordeste, Fortaleza, 2007.
\bibitem{b3} Mattiuz, Ben-Hur. ``Fatores da pré-colheita influenciam a qualidade final dos produtos,'' Visão Agrícola nº7, Revista Visão Agrícola, USP, 2007, pp. 18--21.
\bibitem{b4} De, L.C. and Singh, D.R., ``Post –Harvest Management And Value Addition In Orchids,'' Dynamic Network for Research Works, DNetRW, International Journal of Biological Sciences, IJBS, India, 2016.
\bibitem{b5} L.C. De, Vij SP and Medhi RP, ``Post-Harvest Physiology and Technology in Orchids,'' Journal of Horticulture, India, 2014.
\bibitem{b6} G. Lopez, Roberto and S. Runkle, Erik, ``Environmental Physiology of Growth and Flowering of Orchids,'' Department of Horticulture, Michigan State University, East Lansing, 2005.
\bibitem{b7} De, Lakshman Chandra. ``Good Agricultural Practices of Commercial Orchids, '' Vigyan Varta 1(5): 53-64, 2020.
\bibitem{b8} Alexander Maier, Andrew Sharp and Yuriy Vagapov, ``Comparative Analysis and Practical Implementation
of the ESP32 Microcontroller Module for the Internet of Things,'' School of Applied Science, Computing and Engineering, Glyndwr University, UK, IEEE, 2017.
\bibitem{b9} Gondchawar, Nikesh and S. Kawitkar, Prof. Dr. R., ``IoT based Smart Agriculture,'' International Journal of Advanced Research in Computer and Communication Engineering, 5(6), India, June, 2016.
\bibitem{b10} Vinoth Kumar.P, K.C Ramya, Abishek.J.S, Arundhathy.T.S, Bhavvya.B, Gayathri.V, ``Smart Garden Monitoring and Control System with Sensor Technology,''  Department of Electrical and Electronics Engineering, Sri Krishna College of Engineering and Technology, Kuniamuthur Coimbatore, India, 2021.
\bibitem{b11} Monica M, B.Yeshika, Abhishek G.S, Sanjay H.A, Sankar Dasiga, ``IoT Based Control and Automation of Smart
Irrigation System, An Automated Irrigation System Using Sensors, GSM, Bluetooth and Cloud Technology,'' Proceeding International conference on Recent Innovations is Signal Processing and Embedded Systems, NitteMeenakshi Institute of Technology, Bangalore, India., 2017.
\bibitem{b12} Lauren Murphy, Tosin Alliyu, Andrew Macvean, Mary Beth Kery, Brad A. Myers, ``Preliminary Analysis of REST API Style Guidelines,'' PLATEAU’17, Vancouver, CA, October 23, 2017.
\bibitem{b13} Surwase, Vijay, ``REST API Modeling Languages - A Developer’s Perspective,'' IJSTE - International Journal of Science Technology & Engineering, 2(10), April, 2010.
\bibitem{b14} Ram Naresh Thakur, U. S. Pandey, Jyotir Moy Chatterjee, ``Study of Mobile Application Development using MVC Framework,''Journal of Information Communication Technology and Digital Convergence, 4(2), December, 2019, pp. 13-17. 
\bibitem{b15} Singh, Mr. Siddharth, ``MVC Framework: A Modern Web Application Development Approach and Working,'' International Research Journal of Engineering and Technology (IRJET), 7(1), January, 2020.
\bibitem{b16} Meng Ma, Jun Yang, Ping Wang, Weijie Liu and Jingzhuo Zhang, ``Light-weight and Scalable Hierarchical-MVC
Architecture for Cloud Web Applications,'' IEEE, 2019.
\end{thebibliography}

\end{document}